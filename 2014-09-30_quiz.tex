\documentclass{beamer}

\DeclareMathOperator*{\argmax}{arg\,max}

\usepackage{beamerthemesplit}
\usepackage{tikz}
\usetikzlibrary{backgrounds,fit,shapes.misc}

\usepackage{enumitem}

% Set up citation style
\usepackage{natbib}
\usepackage{bibentry}
\bibpunct{(}{)}{;}{a}{,}{,}
\newcommand{\newcite}[1]{\citet{#1}}
\renewcommand{\cite}[1]{\citep{#1}}

\setbeamercovered{transparent}


\title[LING 402]{Tools and Techniques for \\ Speech and Language Processing}
\author[Tuesday, 30 September 2014]{Lane Schwartz}
\institute[shortinst]{University of Illinois at Urbana-Champaign}

\date{30 Sept 2014}

\begin{document}

% Specify that no bibliography should be printed
\bibliographystyle{plainnat}

\frame{\titlepage}



\section{Chapter 2}

\frame {

Which command prints the name of the current working directory?

\begin{enumerate}[label=\Alph*)]

\item awk

\item dir

\item here

\item man

\item pwd

\end{enumerate}

}


\begin{frame}[fragile]

What does the following command do?



\begin{verbatim}
$ ls /
\end{verbatim}

\begin{enumerate}[label=\Alph*)]

\item Changes the current working directory to the user's home directory

\item Creates a new directory

\item List the contents of the root directory

\item Prints today's date using / as the separator

\end{enumerate}

\end{frame}


\begin{frame}[fragile]

Which of the following is a relative path that refers to the current working directory?

\begin{enumerate}[label=\Alph*)]

\item -

\item /

\item .

\item ..

\item *

\end{enumerate}

\end{frame}


\section{Chapter 4}

\begin{frame}[fragile]

Which of the following commands will delete all files with the suffix .html from the current directory, and only those files?

\begin{enumerate}[label=\Alph*)]

\item ls -l *.html

\item file * .html

\item rm * .html

\item rm *.html

\item touch *.html

\end{enumerate}

\end{frame}


\begin{frame}[fragile]

After running the following command, where will the copied file be located?

\begin{verbatim}
$ cp /usr/include/printf.h ..
\end{verbatim}

\begin{enumerate}[label=\Alph*)]

\item In the current working directory

\item In the parent of the current working directory

\item In the root directory

\item In your home directory

\item In /usr/include

\end{enumerate}

\end{frame}


\section{Chapter 6}

\begin{frame}[fragile]

Is there a difference between these two commands?

\begin{verbatim}
$ ls | head -1
$ ls > temp; head -1 < temp 
\end{verbatim}

\begin{enumerate}[label=\Alph*)]

\item No, they do the same thing

\item The output will be the same, but the side effects will be different

\item In some cases the output will be the same, but the side effects will be different. In other cases the output will be different.

\item Yes, they do completely different things

\end{enumerate}

\end{frame}

\section{Chapter 25}

\begin{frame}[fragile]

What will the following script do?


\begin{verbatim}
#!/bin/bash 

year = $1

mkdir $year

\end{verbatim}



\begin{enumerate}[label=\Alph*)]

\item Create a directory

\item Create a directory, and print a confirmation message

\item Silently fail to create a directory

\item Fail to create a directory, and print a usage message for mkdir

\item I have no idea

\end{enumerate}

\end{frame}


\section{Chapter 32}

\begin{frame}[fragile]

Within a shell script, \$0 provides the name of the shell script, as it was called at the command line.

\ \\ 

What do you think the following command will print?

\begin{verbatim}
$ echo $0
\end{verbatim}

\begin{enumerate}[label=\Alph*)]

\item A blank line

\item An error message

\item Something else

\item I have no idea

\end{enumerate}

\end{frame}


\end{document}
